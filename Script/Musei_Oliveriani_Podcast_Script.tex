\documentclass[hidelinks,12pt,a4paper]{article}
\usepackage[italian]{babel}
\usepackage[utf8]{inputenc}
\usepackage{fourier} 

% To avoid GitHub Action error
\usepackage{hyperref}

% Stop hyphenation
\usepackage[none]{hyphenat}

% Justifying text
\emergencystretch 3em

% Remove first empty page
\usepackage{atbegshi}
\AtBeginDocument{\AtBeginShipoutNext{\AtBeginShipoutDiscard}}

% License
\usepackage[
type={CC},
modifier={by-nc-nd},
version={4.0},
]{doclicense}


\begin{document}
	\begin{flushleft}
		
		\title{\textbf{Copione per il Podcast riguardante i Musei Oliveriani}}
		\author{Alice Balestieri\\Francesco Rombaldoni}
		\date{}
		
		\maketitle
		
		% Adjust page counter
		\setcounter{page}{1}
		\newpage
		
		\tableofcontents
		\newpage
		
		
		\section{Viaggi Nei Musei di Pesaro}
		\subsection{Episodio 1: Notturni Oliveriani}
		\begin{center}
			\textbf{Omamori Group Presenta: "Viaggi nei Musei di Pesaro"}
		\end{center}
		Il nostro scopo è quello di farvi conoscere le bellezze della nostra piccola Pesaro in una chiave innovativa, dal punto di vista di Alice Balestieri, una giovane ragazza ed artista del luogo.\\
		In questo primo episodio vi parleremo di un evento particolare che si tiene nei Musei Oliveriani e si chiama: "Notturni Oliveriani" e viene curato dalla direttrice dell'Ente Olivieri, Brunella Paolini.\\
		In queste serate potrete visitare gratuitamente, durante una speciale apertura notturna i Musei Oliveriani, ma non solo potrete ascoltare musica dal vivo, assistere a piccole performance teatrali, visionare video storici della "Pesaro che fù" proiettati nei cortili della Biblioteca Oliveriana, ed inoltre su prenotazione e pagando un modico biglietto potrete diventare protagonisti di un "Escape Room" dal vivo nelle stanze della Biblioteca Oliveriana, il tutto fino all'alba!.\\
		I visitatori del museo saranno accompagnati da una guida nella visione delle varie sale del museo e potranno vivere la magica atmosfera alla: "Una Notte al Museo" in cui i reperti sembrano quasi prendere vita tanta è l'emozione di vederli così da vicino sotto le notturne luci soffuse del museo, che ti faranno sentire un po' come un giovane archeologo alla "Indiana Jones" o alla "Tomb Raider" che si avventura alla scoperta di antiche tombe.\\
		All'interno delle sale del museo potrai infatti scoprire che particolari oggetti facessero parte del corredo funebre delle persone dell'età del ferro e come esso fosse composto.\\
		Pensa a come questi oggetti dell'età del ferro, come in un viaggio con una macchina del tempo,ci mostrino gli usi e i costumi che erano soliti avere le persone di quell'epoca.\\
		Se osserverai bene inoltre avrai la possibilità di scorgere dei piccoli cassetti in basso, ed estraendoli potrai osservare alcuni  reperti ancor più da vicino ed avere la sensazione un po' come se a ritrovarli fossi stato proprio tu!.\\
		Se ti è piaciuto questo Podcast segui il nostro canale o visita la nostra pagina Github: "Pomodoro Musei di Pesaro" per vedere altre curiosità e rimanere aggiornato.\\
		Se hai vissuto anche tù questa esperienza faccelo sapere nei commenti in basso.\\
		Prima di chiudere questo primo episodio voglio ringraziare il mio collaboratore Francesco Rombaldoni che ha curato le musiche ed il montaggio di queste registrazioni.\\
		Il racconto di questa esperienza continuerà nell'episodio successivo, per cui se vorrete conoscerne il finale continuate a seguirci!.\\
		
		
		\subsection{Episodio 2: Escape Room agli Oliveriani seconda parte dell'Episodio Notturni Oliveriani}
		\begin{center}
			\textbf{Omamori Group Presenta: "Viaggi nei Musei di Pesaro"}
		\end{center}
		In questo secondo episodio, come promesso, concluderò di raccontarvi l'esperienza dei "Notturni Oliveriani", parlandovi in particolare di come ho vissuto insieme ad un gruppo di amici, l'approccio giocoso nel comprendere meglio i reperti presenti nelle varie stanze della Biblioteca Oliveriana attraverso la partecipazione attiva dell'attività della "Escape Room" dal vivo organizzata a cura dell'Enigma Escape Room.\\
		Durante l'attività della "Escape Room" attraverso la risoluzione di vari enigmi differenti tra loro, creati per voi dall'antico fondatore dell'Ente Olivieri:" Annibale degli Abbati Olivieri", divertendoti a : cercare chiavi, indizi sparsi nelle stanze, aprendo lucchetti ,decifrando le risposte all'interno di alcuni mini-giochi e scoprendo la misteriosa parola per aprire il cryptex ti rimarranno sicuramente molto più impresse senza accorgertene le informazioni sul Museo e Biblioteca Oliveriana che hai appreso durante la serata e se avrai completato correttamente la "Escape Room" riuscendo ad uscire dalla Biblioteca e concludendo così il gioco potrai ricevere come premio all'esterno nel cortile della Biblioteca, se sei maggiorenne un'assaggio di vino o altrimenti ti verranno offerti alcuni deliziosi e gustosi dolcetti artigianali.\\
		Queste serate dei "Notturni Oliveriani" sono da non perdere perchè ti danno modo di godere di un approccio più rilassato al mondo culturale e all'insegna del divertimento e ti permetteranno se visiti Pesaro per la prima volta di imparare in allegria la storia della nostra città, mentre se sei residente in essa potrai riscoprire il piacere di viverci vedendola in una nuova ottica.\\
		Vi voglio ricordare nuovamente se vi è piaciuto questo podcast di seguire il nostro canale o la nostra pagina Github: "Pomodoro Musei di Pesaro" per accedere a nuove informazioni e rimanere aggiornati.\\
		Se avete partecipato all'attività della "Escape Room", raccontateci com'è andata nei commenti in basso.\\
		Rinnovo i miei ringraziamenti a Francesco Rombaldoni che dietro le quinte cura la musica e il montaggio delle registrazioni senza il quale non potreste ascoltare la mia voce.\\
		Nel prossimo episodio vi spiegheremo le curiosità che si nascondono dietro alle incisioni presenti nella Biblioteca Oliveriana, per cui rimanete sintonizzati sul nostro canale!.\\
		
		\vspace*{\fill}
		% Print license shield
		\doclicenseThis
	\end{flushleft}
\end{document}
