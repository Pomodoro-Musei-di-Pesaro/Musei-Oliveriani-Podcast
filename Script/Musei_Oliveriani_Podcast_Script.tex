\documentclass[hidelinks,12pt,a4paper]{article}
\usepackage[italian]{babel}
\usepackage[utf8]{inputenc}
\usepackage{fourier} 

% To avoid GitHub Action error
\usepackage{hyperref}

% Stop hyphenation
\usepackage[none]{hyphenat}

% Justifying text
\emergencystretch 3em

% Remove first empty page
\usepackage{atbegshi}
\AtBeginDocument{\AtBeginShipoutNext{\AtBeginShipoutDiscard}}

% License
\usepackage[
type={CC},
modifier={by-nc-nd},
version={4.0},
]{doclicense}


\begin{document}
	\begin{flushleft}
		
		\title{\textbf{Copione per il Podcast riguardante i Musei Oliveriani}}
		\author{Alice Balestieri\\Francesco Rombaldoni}
		\date{}
		
		\maketitle
		
		% Adjust page counter
		\setcounter{page}{1}
		\newpage
		
		\tableofcontents
		\newpage
		
		
		\section{Viaggi Nei Musei di Pesaro}
		\subsection{Episodio 1: Notturni Oliveriani}
		\begin{center}
			\textbf{Omamori Group Presenta: "Viaggi nei Musei di Pesaro"}
		\end{center}
		Il nostro scopo è quello di farvi conoscere le bellezze della nostra piccola Pesaro in una chiave innovativa, dal punto di vista di Alice Balestieri, una giovane ragazza ed artista del luogo.\\
		In questo primo episodio vi parleremo di un evento particolare che si tiene nei Musei Oliveriani e si chiama: "Notturni Oliveriani" e viene curato dalla direttrice dell'Ente Olivieri, Brunella Paolini.\\
		In queste serate potrete visitare gratuitamente, durante una speciale apertura notturna i Musei Oliveriani, ma non solo potrete ascoltare musica dal vivo, assistere a piccole performance teatrali, visionare video storici della "Pesaro che fù" proiettati nei cortili della Biblioteca Oliveriana, ed inoltre su prenotazione e pagando un modico biglietto potrete diventare protagonisti di un "Escape Room" dal vivo nelle stanze della Biblioteca Oliveriana, il tutto fino all'alba!.\\
		I visitatori del museo saranno accompagnati da una guida nella visione delle varie sale del museo e potranno vivere la magica atmosfera alla: "Una Notte al Museo" in cui i reperti sembrano quasi prendere vita tanta è l'emozione di vederli così da vicino sotto le notturne luci soffuse del museo, che ti faranno sentire un po' come un giovane archeologo alla "Indiana Jones" o alla "Tomb Raider" che si avventura alla scoperta di antiche tombe.\\
		All'interno delle sale del museo potrai infatti scoprire che particolari oggetti facessero parte del corredo funebre delle persone dell'età del ferro e come esso fosse composto.\\
		Pensa a come questi oggetti dell'età del ferro, come in un viaggio con una macchina del tempo,ci mostrino gli usi e i costumi che erano soliti avere le persone di quell'epoca.\\
		Se osserverai bene inoltre avrai la possibilità di scorgere dei piccoli cassetti in basso, ed estraendoli potrai osservare alcuni  reperti ancor più da vicino ed avere la sensazione un po' come se a ritrovarli fossi stato proprio tu!.\\
		Se ti è piaciuto questo Podcast segui il nostro canale o visita la nostra pagina Github: "Pomodoro Musei di Pesaro" per vedere altre curiosità e rimanere aggiornato.\\
		Se hai vissuto anche tù questa esperienza faccelo sapere nei commenti in basso.\\
		Prima di chiudere questo primo episodio voglio ringraziare il mio collaboratore Francesco Rombaldoni che ha curato le musiche ed il montaggio di queste registrazioni.\\
		Il racconto di questa esperienza continuerà nell'episodio successivo, per cui se vorrete conoscerne il finale continuate a seguirci!.\\
		
		
		\subsection{Episodio 2: Escape Room agli Oliveriani seconda parte dell'Episodio Notturni Oliveriani}
		\begin{center}
			\textbf{Omamori Group Presenta: "Viaggi nei Musei di Pesaro"}
		\end{center}
		In questo secondo episodio, come promesso, concluderò di raccontarvi l'esperienza dei "Notturni Oliveriani", parlandovi in particolare di come ho vissuto insieme ad un gruppo di amici, l'approccio giocoso nel comprendere meglio i reperti presenti nelle varie stanze della Biblioteca Oliveriana attraverso la partecipazione attiva dell'attività della "Escape Room" dal vivo organizzata a cura dell'Enigma Escape Room.\\
		Durante l'attività della "Escape Room" attraverso la risoluzione di vari enigmi differenti tra loro, creati per voi dall'antico fondatore dell'Ente Olivieri:" Annibale degli Abbati Olivieri", divertendoti a : cercare chiavi, indizi sparsi nelle stanze, aprendo lucchetti ,decifrando le risposte all'interno di alcuni mini-giochi e scoprendo la misteriosa parola per aprire il cryptex ti rimarranno sicuramente molto più impresse senza accorgertene le informazioni sul Museo e Biblioteca Oliveriana che hai appreso durante la serata e se avrai completato correttamente la "Escape Room" riuscendo ad uscire dalla Biblioteca e concludendo così il gioco potrai ricevere come premio all'esterno nel cortile della Biblioteca, se sei maggiorenne un'assaggio di vino o altrimenti ti verranno offerti alcuni deliziosi e gustosi dolcetti artigianali.\\
		Queste serate dei "Notturni Oliveriani" sono da non perdere perchè ti danno modo di godere di un approccio più rilassato al mondo culturale e all'insegna del divertimento e ti permetteranno se visiti Pesaro per la prima volta di imparare in allegria la storia della nostra città, mentre se sei residente in essa potrai riscoprire il piacere di viverci vedendola in una nuova ottica.\\
		Vi voglio ricordare nuovamente se vi è piaciuto questo podcast di seguire il nostro canale o la nostra pagina Github: "Pomodoro Musei di Pesaro" per accedere a nuove informazioni e rimanere aggiornati.\\
		Se avete partecipato all'attività della "Escape Room", raccontateci com'è andata nei commenti in basso.\\
		Rinnovo i miei ringraziamenti a Francesco Rombaldoni che dietro le quinte cura la musica e il montaggio delle registrazioni senza il quale non potreste ascoltare la mia voce.\\
		Nel prossimo episodio vi spiegheremo le curiosità che si nascondono dietro alle incisioni presenti nella Biblioteca Oliveriana, per cui rimanete sintonizzati sul nostro canale!.\\
		
		\subsection{Episodio 3: Le curiosità nascoste dietro alle incisioni ai Musei Oliveriani}
		\begin{center}
			\textbf{Omamori Group Presenta: "Viaggi nei Musei di Pesaro"}
		\end{center}
		In questo terzo episodio, come annunciato alla fine del precedente vi parlerò delle curiosità nascoste dietro alle incisioni di Michele Artazù che tramandò questa tecnica ai propri figli, Luigi e Saverio, essi si esprimevano con maniacale cura in una tecnica di stampa mista all'uso del collages , talvolta con parti a colori e rifiniture in pizzo ed oro. Il cognome singolare dei due gli è stato dato per via della loro provenienza dall'omonimo comune spagnolo, situato nella comunità autonoma della Navarra.\\
		Questi due incisori inizialmente facevano parte di una comunità gesuita che per un periodo  di tempo si stabilì a Pesaro.\\
		Il padre fu incisore per l'architetto Carducci e presso la Tipografia Gavelli, mentre il figlio Saverio lavorò come calligrafo per il Tipografo comunale Annesio Nobili.\\
		La serie di pregiate inisioni presenti all'ingresso della Biblioteca Oliveriana possono testimoniare le conoscenze geografiche dall'epoca, vediamo infatti qui rappresentati i quattro continenti di cui si era a conoscenza:"Africa, Europa, America ed Asia.\\
		Tra le incisioni che potete ammirare vi parlerò in particolar modo di due di esse, che ritraggono il progetto originale di come doveva apparire la Basilica di Santa Maria Assunta di Pesaro, oggi meglio conosciuta come Duomo di Pesaro.\\
		Il Duomo di Pesaro è un luogo che a prescindere dalla vostra religione vi consiglio di visitare perchè è un piccolo gioiello artistico, ciò per cui è più caratteristico è la presenza di numerosi mosaici, oggi visibili al pubblico tramite delle vetrate incastonate all'interno della pavimentazione, ma non tutti sanno che come riportano le incisioni di Michele e Luigi Artazù all'esterno il Duomo era delimitato da una serie di colonne risalenti all'epoca romana, di ordine ionico e corinzio con dipinte variopinte decorazioni, i cui resti si trovano ora all'interno dei cortili degli Oliveriani, altra testimonianza ancora visibile  della presenza romana sono il rosone e gli spioventi di quest'epoca che ancora fanno parte del Duomo.\\
		Nel 1700 il fondatore dell'Ente Olivieri, Annibale degli Abbati Olivieri ritrovò lui stesso delle nuove porzioni degli antichi mosaici presenti nella Bailica.\\
		Ciò che venne ritrovato da Olivieri furono una serie di pregiati mosaici alcuni di essi risaenti tra il decimo e il tredicesimo secolo mentre altri appartenenti al periodo tardo antico che introducevano nuove raffigurazioni di origine medievale di cui possiamo osservare la serie di creature fantastiche come: "Lamie, malvagi spiriti notturni, una sirena, il Centauro-Saggittario ed un Tritone"; queste figure erano state scelte perchè considerate dal contenuto "esemplare" ed erano ispirate da Enciclopedie, Bestiari e Romanzi che circolavano in Europa all'epoca ed alcune iscrizioni contenute nei mosaici ci mostrano ancora oggi i nomi dei committenti che offrirono il compenso per la realizzazione di questi splendidi mosaici che sono emblematici di come ci fosse una stratificazione multiculturale della visione della religione all'epoca.\\
		Il mosaico più iconico a cui si associa Pesaro appartiene proprio ad uno di questi e raffigura una Sirena Bicaudata, cioè raffigurata con due code.\\
		I mosaici ritrovati da Annibale degli Abbati Olivieri furono però rinterrati perchè al tempo non si sapeva ancora la tecnica per poter mantenere intatta la loro bellezza, nonostante ci fosse già la voglia di renderli fruibili a tutti. Nel 1866  Giovan Battista Carducci famoso architetto Pesarese pubblicò ulteriori testi di cui fece commissionare le incisioni a Michele e Luigi Artazù ,che ritraggono con minuzia sia come doveva apparire all'epoca sia l'esterno della Chiesa che la pavimentazione interna che mettevano in luce ulteriori ritrovamenti musivi ma anche questa volta vennero risotterrati, per permettere a Carducci che era stato incaricato di ristrutturare il Duomo di poter rifare la pavimentazione a suo piacimento.\\
		Finalmente nel diciannovesimo secolo parte dei mosaici fu aperta alla visione del pubblico, i più antichi presenti in questa Chiesa sono risalenti addirittura all'epoca Paleocristiana e documentano come la Chiesa fosse originariamente un luogo di culto di origini pagane, ma i mosaici non sono l'unica attrattiva della Basilica, sono presenti al suo interno anche pregiate opere pittoriche ed affreschi che invito i nostri ascoltatori a visitare.\\
		Se siete curiosi di ricevere ulteriori approfondimenti sulla storia del Duomo di Pesaro o ci volete raccontare la vostra esperienza dopo aver visitato questo luogo ricco di arte fatecelo sapere nei commenti in basso!\\
		Per aggiornamenti vi invito come sempre a seguire il nostro canale e la nostra pagina Github: "Pomodoro Musei di Pesaro" e vi anticipo che la prossima puntata riguarderà la Stele di Novilara e le sue contoversie per cui siate curiosi e continuate a seguirci!.Per concludere ringrazio come sempre perla sua collaborazione Francesco Rombaldoni che mi supporta per l'elaborazione delle musiche e il montaggio della registrazione di questa serie di Podcast.\\
		(parte delle informazioni storico- artistiche sono state tratte dalla pagina google, arcidiocesipesaro.it e poi da mè rielaborate).\\
		
		\subsection{Episodio 4: La Stele di Novilara e le sue Controversie}
		\begin{center}
			\textbf{Omamori Group Presenta: "Viaggi nei Musei di Pesaro"}
		\end{center}
		Come annunciato nell'episodio precedente oggi tratteremo le controversie sulle quali ancora oggi gli archeologi e storici tengono accesi dibattiti riguardanti la famosa "Stele di Novilara", ora conservata all'interno dei Musei Oliveriani, essa per alcuni potrebbe essere un falso storico appartenente al 900, come poi è stato appurato che siano altre steli che fanno parte del museo.\\
		Io sostengo come altri, che invece questa stele sia un pezzo originale, la Stele di Novilara o Stele di Neumachia è infatti un celebre esempio di arte Picena appartenente all'incirca tra il sesto e il settimo secolo avanti Cristo e serviva a raffigurare ai posteri le imprese di un soldato che aveva preso parte a una gloriosa guerra avvenuta via mare e le circostanze in cui esso era morto.\\
		Io sostengo sia un'originale questa stele anche in vista di parte delle raffigurazioni di carattere erotico presenti in essa e per via della semplice e per noi un po`rozza maniera in cui essa è stata scolpita su un blocco di pietra arenaria, dubito che un falso storico avrebbe presentato questa tipologia di scene e sarebbe stato realizzato nel 900 con questo stile.\\
		La Stele di Novilara presenta inoltre come potete notare creature mitiche che si credevano esistenti all'epoca come:"Serpenti, Mostri e Draghi marini".\\
		La Stele di Novilara ci mostra inoltre le correnti marine raffigurate simbolicamente come una sorta di frecce e come nelle polene delle navi fossero presenti a loro volta delle raffigurazioni di draghi , probabilmente per esorcizzre la paura che i marinai e gli schiavi che remavano in queste navi avessero per queste creature fantastiche.\\
		I guerrieri tra i quali il nostro defunto protagonista a cui era dedicata la strele, che narrava le sue imprese ai posteri, si può notare che combattevano con una sorta di lance appuntite tra loro.\\
		Nell'atrio che precede l'ingresso alla Biblioteca Oliveriana possiamo notare una ricostruzione tramite modellino di come doveva apparire la nave più grande che occupa la parte centrale della Stele di Novilara.\\ 
		Le controversie legate a questa stele sono frutto principalmente dalle iscrizioni sul suo retro in un linguaggio dalla natura ambigua la cui interpretazione non è ancora del tutto chiara.\\
		E voi cosa ne pensate, secondo voi è un falso storico o un pezzo originale? fatecelo sapere  nei commenti in basso!\\
		Vi ricordo di continuare a seguire il nostro canale o la nostra pagina Github:"Pomodoro Musei di Pesaro"per avere ulteriori aggiornamenti.\\
		Per concludere come sempre rinnovo i miei ringraziamenti per il supporto che mi dà nel montaggio delle registrazioni e nella realizzazione delle musiche da parte di Francesco Rombaldoni. Nella prossima puntata parleremo di come fatto l'Anemoscopio, se siete curiosi su cosa fosse e a cosa servisse mi raccomando continuate a seguirci!.\\
		
		\subsection{Episodio 5: Come era fatto l'Anemoscopio di Boscovitch?}
		\begin{center}
			\textbf{Omamori Group Presenta: "Viaggi nei Musei di Pesaro"}
		\end{center}
		Come vi ho accennato alla fine del precedente episodio, oggi vi parlerò di come era fatto l'Anemoscopio di Boscovitch e a cosa esso servisse.\\
		L' Anemoscopio presente nella prima sala dei musei Oliveriani fu scoperto in maniera fortuita tramite degli scavi effettuati da dei contadini per realizzare una vigna lungo la Via Appia fuori da Porta Capena a Roma.\\
		Il reperto archeologico in pietra non è particolarmente grande ma presenta come l'Anemoscopio vaticano un foro al centro nel quale era posta un'asta per posizionare una banderuola e ai cui lati della ruota erano presenti altri fori più piccoli per posizionare delle asticelle in bronzo che dovevano servire a determinare tramite la posizione del movimento della bandiera , quale vento tra i dodici indicati stesse soffiando. Nella parte superiore dell'Anemoscopio è raffigurato il planisfero, mentre nel suo spessore sono riportati prima in greco e poi tradotti in latino, probabilmente da Plinio il Vecchio i nomi dei dodici venti, questo strumento meteorologico doveva probabilmente servire durante i viaggi commerciali di tipologia navale per determinare con certezza quale vento stesse soffiando, che come è noto è determinante saper comprendere i venti per poter adeguare le vele di una nave durante il tragitto di questi viaggi.\\
		Il sistema tramite i quali sono riportati dodici venti, come nell'Anemoscopio pesarese si dice fosse uno dei preferiti da Seneca, ma non in tutti i reperti veniva utilizzato, è interessante infatti notare le differenze date dagli studi e dall'applicazione pratica che venivano riportati nella costruzione degli Anemoscopi di luogo in luogo, alcuni per esempio riportano solamente otto venti, altri rappresentavano i venti in forma antropomorfa tramite alcune statue poste ai lati della superficie dell'anemoscopio, mentre altri ancora riportavano anche la posizione di alcuni astri.\\
		L'Anemoscopio presente nel Museo Oliveriano risale al lontano secondo secolo dopo Cristo ed è stato realizzato scolpendo nella forma di un disco il marmo lunense, poi purtroppo rotto per via del tempo in due pezzi perfettamente combacianti, nella sua faccia superiore sono tracciati sei diametri, i cui estremi terminano con piccoli fori in cui erano presenti le asticelle in bronzo, quattro di essi raggiungono gli estremi di quattro parallele che indicavano i tropici e i circoli polari, disposti rispettivamente due sopra e due sotto ad un diametro orizzontale che indicava invece la linea degli equinozi, mentre il sesto diametro indicava il meridiano celeste ed era stato tracciato perpendicolarmente all'ultimo.\\
		L'Anemoscopio di Pesaro passò di mano in mano, inizialmente acquistato da Francescesco Alfano, un antiquario di Roma nel 1759 da dei contadini che lo avevano rinvenuto poi venduto all'archeologo Paolo Maria Paciaudi e poi da lui regalato ad Annibale degli Abbati Olivieri perchè divenisse un'importante pezzo della sua collezione archeologica museale e così in effetti è stato come potete notare anche voi visitatori.\\
		E voi che anemoscopi avete avuto occasione di vedere nei vostri viaggi, ditecelo nei commenti sotto!\\
		Vi ricordo nuovamente di seguire il nostro canale e la nostra pagina Github:"Pomodoro Musei di Pesaro" e rinnovo nuovamente i miei ringraziamenti a Francesco Rombaldoni che si occupa delle musiche e del montaggio video di questa serie di Podcast.\\
		Continuate a seguirci  mi raccomando, perché nel prossimo episodio vi porteremo indietro nel tempo in viaggio nell'Antica Roma.\\
		
		\subsection{Episodio 6: Viaggio nel passato alla scoperta dell'Antica Roma}
		\begin{center}
			\textbf{Omamori Group Presenta: "Viaggi nei Musei di Pesaro"}
		\end{center}
		Quindi allacciate le cinture e preparatevi a tornare indietro nel tempo con noi!.\\
		All'interno delle sale dei Musei Oliveriani potrete notare una serie di busti in pietra femminili e maschili, che ci mostrano le elaborate acconciature di capelli e i canoni di bellezza della moda dell'epoca.\\ 
		Proseguendo lungo il percorso, potrete ammirare i colorati mosaici di crature marine che abbellivano i pavimenti interni alle "Domus", ovvero le case appartenenti alla nobiltà romana, perchè ovviamente, i ceti più bassi non potevano permettersi di abbellire le loro case con pavimenti mosaicati dai colori sgargianti e anche le loro vesti rispetto a quelle nobiliari avrebbero dovuto avere dei colori più scialbi, probabilmente tendenti al marrone o al grigio, mentre quelle nobiliari potevano essere:"bianco alba che era un colore deciso ma opaco, rosso porpora (che si ricavava dall'interno di alcuni crostacei), arancio o violacee, intessute d'oro o d'argento, con inserti ricamati in seta o intessute in lana", anche il colore delle vesti differenziava le diverse cariche o mansioni degli abitanti romani.\\
		Possiamo inoltre notare come anche oggetti di uso comune come le "lucerne", cioè le lampade ad olio in avorio o coccio, fossero finemente decorate tanto da sembrare delle "lampade magiche" e forse in un certo senso dovevano esserlo in un mondo dove la sera senza di esse doveva essere veramente molto buio, infatti tutta la vita delle persone e i loro lavori si basavano sull'utilizzo delle ore di luce all'interno di una giornata, erano quindi di certo ritmi più naturali rispetto ai nostri.\\
		A fine Ottocento però lo studioso Henrich Dressel, in visita a Pesaro per studiare queste "lucerne" mise in dubbio buona parte di questa collezione, considerandole dei falsi cosa ancor oggi dubbia.\\
		I Romani inoltre davano particolare importanza alla loro igiene personale, come sappiamo infatti furono inventori di un complesso sistema di acquedotti sotterranei e delle terme, ma oggi non parleremo di questo, ma delle loro vasche in pietra con scolpite in rilievo decorazioni con delfini o piante nelle quali, seppur a noi sembrino ora piccole, doveva essere per loro un comodo lusso potersi fare un bagno.\\
		All'interno delle sale e del cortile dei Musei Oliveriani possiamo ammirare inoltre resti di antiche colonne romane, alcune delle quali come già detto nelle puntate precedenti, appartenenti al Duomo di Pesaro.\\
		E voi siete affascinati dalla avanzata società romana e dalle sue conoscenze? se la risposta è sì non potete non visitare il Museo Oliveriano! \\
		Se siete venuti qui in visita, fatecelo sapere nei commenti in basso.\\
		Come sempre ringrazio Francesco Rombaldoni per la sua collaborazione per il montaggio e l'elaborazione delle musiche di questo Podcast e non lasciateci perché nella prossima puntata continueremo a parlare di Antica Roma con : "La Leggenda del Lucus Pisaurensis", se non sapete cosa fosse, rimanete sintonizzati sul nostro canale!\\
		
		\subsection{Episodio 7: La Leggenda del Lucus Pisaurensis}
		\begin{center}
			\textbf{Omamori Group Presenta: "Viaggi nei Musei di Pesaro"}
		\end{center}
		Oggi vi racconterò la leggenda del Lucus Pisaurensis, luogo di culto pagano che Passeri sosteneva di aver individuato nei dintorni della nostra attuale "Santa Veneranda", più precisamente vicino a un misterioso boschetto.\\
		Non sappiamo se questo luogo leggendario sia reale, ma possiamo dire con certezza che all'interno del nostro territorio un luogo di culto pagano sicuramente c'era, poiché ci sono arrivati innumerevoli reperti a testimonianza di ciò, ma voi vi chiederete come fosse un luogo di culto pagano, ma non vi preoccupate, se avrete solamente un po' di pazienza andremo subito a spiegarvelo.\\
		Un luogo di culto pagano innanzitutto doveva presentare vari lastroni in pietra con incisi i nomi e il voto di preghiera o miracolo che si voleva che la divinità compisse e noi possiamo vederne diversi esempi grandi e piccoli all'interno delle sale dei Musei Oliveriani.\\
		I romani inoltre dediti a queste preghiere nei luoghi di culto pagani, erano soliti realizzare delle statuette votive per dare maggiore forza e concretezza alle loro richieste per le divinità.\\ 
		Gli uomini chiedevano spesso l'aiuto per una maggiore virilità e una stirpe di sesso maschile, ma anche che gli venissero curati malanni a piedi o braccia, oppure una maggior proliferazione di raccolto o bestiame.\\
		Le donne chiedevano invece di poter dare alla luce bambini o bambine sane, anch'esse la cura di certe malattie che le affliggevano, ma anche la grazia che il proprio consorte trovasse fortuna e ricchezze nella propria carriera lavorativa permettendo così ad esse e alla stirpe familiare un futuro più sereno ed agiato.\\
		Purtroppo non sempre le divinità erano favorevoli nel compiere tali preghiere, infatti possiamo notare all'interno del Museo Oliveriano come una triste sorte si abbattesse su alcune famiglie, private dei loro figli in tenera età di cui possiamo vedere alcuni inquietanti ma rari scheletri interi, il cui corredo funebre consisteva in una serie di giocattoli che il bambino o bambina poteva portare con sè nel suo viaggio verso l'aldilà.\\
		E voi credete nell'influsso divino e nella buona o cattiva sorte? Fatecelo sapere nei commenti in basso.\\
		Rinnovo come sempre i miei ringraziamenti a Francesco Rombaldoni nella realizzazione delle musiche e nel montaggio delle registrazioni per questo Podcast.\\
		Continuate a seguirci, perchè con il prossimo episodio vi parleremo di quelle che potevano essere nell'epoca romana, "gli antenati delle Action figures".\\
		
		\section{Episodio 8: Gli antenati delle Action Figures}
		\begin{center}
			\textbf{Omamori Group Presenta: "Viaggi nei Musei di Pesaro"}
		\end{center}
		Oggi vi parleremo di come allora come oggi, piacesse decorare la propria casa con statuette, infatti come a noi oggi piacciono tanto action- figures o i FUNKO-POP,anche ai romani piaceva decorare le proprie "domus" o abitazioni che dir si voglia con statuette.\\
		Per noi oggi sono statuette che associamo alla cosiddetta "Cultura Pop" alla quale appartengono anime, manga, graphic novels, fumetti e videogiochi, mentre i romani decoravano la propria casa con queste statuette bronzee o in stagno più o meno realistiche o realizzate da artigiani più o meno abili perché esse rappresentavano i "Lari", ovvero le divinità della casa e del focolare domestico, oppure invece erano semplicemente state scelte come decorazione artistica dai nobili dell'epoca.\\
		I Lari o le statuette artistiche potevano essere di svariate tipologie: potevano rappresentare Venere o Minerva, oppure delle figure maschili a cavallo, dedite alla caccia o coraggiosi guerrieri con lance , elmi e scudi dalla forma rotonda probabilmente scudi detti Aspis o Hòplon.\\
		Di alcune statue possiamo osservare il volto dai grandi occhi sbarrati ed il naso marcato.\\
		Le statuette potevano però essere anche la raffigurazione artistica di alcuni animali, cani, papere, buoi, cinghiali ed altro ancora, queste servivano probabilmente a celebrare la fedeltà dei propri animali domestici, oppure a augurare una buona battuta di caccia.\\
		Le statue però non erano tutte di carattere ludico o soggetti di culto votivo, alcune di esse erano infatti integrate come decorazioni di oggetti di uso comune, quali coppe, vasi, piatti, cucchiai,bilancieri, vassoi e tanto altro.\\
		E voi avete action- figures o Funko-pop  che decorano la vostra camera? O avete altre decorazioni artistiche?\\
		Se si quali?, siamo curiosi di saperlo, per cui se volete ditecelo nei commenti in basso!\\
		Rinnovo i miei ringraziamenti a Francesco Rombaldoni che come sempre mi dà il suo supporto nella creazione delle musiche e nel montaggio di queste registrazioni.\\
		Continuate a seguirci perché nella prossima puntata parleremo dei "Grandi Guerrieri Piceni del Settimo secolo a. C."\\
		
		\subsection{Episodio 9: I Grandi Guerrieri Piceni del Settimo secolo a.C.}
		\begin{center}
			\textbf{Omamori Group Presenta: "Viaggi nei Musei di Pesaro"}
		\end{center}
		
		\vspace*{\fill}
		% Print license shield
		\doclicenseThis
	\end{flushleft}
\end{document}
