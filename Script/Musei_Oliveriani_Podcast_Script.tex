\documentclass[hidelinks,12pt,a4paper]{article}
\u\begin{flushright}
	sepackage[italian]{babel}
\usepackage[utf8]{inputenc}

\end{flushright}\usepackage{fourier} 

% Stop hyphenation
\usepackage[none]{hyphenat}

% Justifying text
\emergencystretch 3em

% Remove first empty page
\usepackage{atbegshi}
\AtBeginDocument{\AtBeginShipoutNext{\AtBeginShipoutDiscard}}

% License
\usepackage[
type={CC},
modifier={by-nc-nd},
version={4.0},
]{doclicense}


\begin{document}
	\begin{flushleft}
		
		\title{\textbf{Copione per il Podcast riguardante i Musei Oliveriani}}
		\author{Alice Balestieri\\Francesco Rombaldoni}
		\date{}
		
		\maketitle {\scshape\bfseries Viaggi Nei Musei di Pesaro:\\}
		\item Omamori Group Presenta: "Viaggi nei Musei di Pesaro".
		      Il nostro scopo è quello di farvi conoscere le bellezze della nostra piccola Pesaro in una chiave innovativa, dal punto di vista di Alice Balestieri, una giovane ragazza ed artista del luogo.
		      In questo primo episodio vi parleremo di un'evento particolare che si tiene nei Musei Oliveriani e si chiama: "Notturni Oliveriani" e viene curato dalla direttrice dell'Ente Olivieri, Brunella Paolini.
		      In queste serate potrete visitare gratuitamente, durante una speciale apertura notturna i Musei Oliveriani, ma non solo potrete ascoltare musica dal vivo, assistere a piccole performance teatrali, visionare video storici della "Pesaro che fù" proiettati nei cortili della Biblioteca Oliveriana, ed inoltre su prenotazione e pagando un modico biglietto potrete diventare protagonisti di un'Escape Room dal vivo nelle stanze della Biblioteca Oliveriana, il tutto fino all'alba!.
		      I visitatori del museo saranno accompagnati da una guida nella visione delle varie sale del museo e potranno vivere la magica atmosfera alla: "Una Notte al Museo"in cui i reperti sembrano quasi prendere vita tanta è l'emozione di vederli così da vicino sotto le notturne luci soffuse del museo, che ti faranno sentire un po'come un giovane archeologo alla "Indiana Jones" o alla "Tomb Raider" che si avventura alla scoperta di antiche tombe.
		      All'interno delle sale del museo potrai infatti scoprire che particolari oggetti facessero parte del corredo funebre delle persone dell'età del ferro e come esso fosse composto.
		      Pensa a come questi oggetti dell'età del ferro, come in un viaggio con una macchina del tempo,ci mostrino gli usi e i costumi che erano soliti avere le persone di quell'epoca.
		      Se osserverai bene inoltre avrai la possibilità di scorgere dei piccoli cassetti in basso, ed estraendoli potrai osservare alcuni  reperti ancor più da vicino ed avere la sensazione un'po'come se a ritrovarli fossi stato proprio tu!.
		      Se ti è piaciuto questo Podcast segui il nostro canale o visita la nostra pagina Github: "Pomodoro Musei di Pesaro" per vedere altre curiosità e rimanere aggiornato.
		      Se hai vissuto anche tù questa esperienza faccelo sapere nei commenti in basso.
		      Prima di chiudere questo primo episodio voglio ringraziare il mio collaboratore Francesco Rombaldoni che ha curato le musiche ed il montaggio di queste registrazioni.
		      Il racconto di questa esperienza continuerà nell'episodio successivo, per cui se vorrete conoscerne il finale continuate a seguirci!.
		      
		
		% Adjust page counter
		\setcounter{page}{1}
		\newpage
		
		\tableofcontents
		\newpage
		
		
		% ---------- Insert Text In this space. ----------
		
		\vspace*{\fill}
		% Print license shield
		\doclicenseThis
	\end{flushleft}
\end{document}
		
		